\section{Vestiario}

Come ci si veste? Vanno valutati differenti aspetti: l'ambiente, a chi sto
parlando e di nuovo gli obiettivi.

In una situazione di public speaking di tipo business (ma anche pitch
universitari) l'obiettivo è lasciare un impressione positiva agli
interlocutori, in modo tale da agevolare il resto della comunicazione. Partendo
dal presupposto che, come già accennato precedentemente, l'abito fa il
monaco\footnote{Ovvero il contenitore influenza molto il contenuto.}, il
vestiario conta e influenza come una persona appare. La valutazione di oggetti,
idee, persone viene fatta molto velocemente dall'essere umano, ed è una sua
propietà particolare. L'abilità di essere in grado di eseguire valutazioni in
poco tempo porta al fatto che valutazioni di presentazioni avvengono nel 70\%
dei casi nei primi 30secondi. L'aspetto dell'individuo in una presentazione è
un aspetto consistente, ma gioca un ruolo importante anche la comunicazione non
verbale.

L'estetica di un individuo, quindi, è abbastanza importante. Bisogna sfruttare
questo meccanismo a proprio favore, dato che questa meccanica si applica a tutti
gli umani.

Un primo passo è valutare se stessi e il pubblico. Se stessi perché bisogna
sentirsi a proprio agio, e non c'è niente di peggio che, per esempio,
presentarsi ad un colloquio vestiti da prima comunione. Questo comunque non
significa essere pigri.

In un colloquio di lavoro, è opportuno vestirsi adeguatamente al ruolo che si
vuole andare a ricoprire. Anche vestirsi nella direzione che si vuole ricoprire
è un segnale alla candidatura che si vuole andare ad avere: abbigliandosi in
una certa maniera comunica implicitamente la volontà a riguardo. È sempre bene
ricordarsi che è bene controllare l'abito che si veste, non il contrario.
