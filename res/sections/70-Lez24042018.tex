\section{Vestiario}

Come ci si veste? Devono essere valutati differenti aspetti: sé stessi,
l'ambiente, a chi sto parlando e gli obiettivi a cui si punta.

Se l'obiettivo a cui puntiamo è solamente inserirsi in un contesto va bene
uniformarsi agli altri, altrimenti è necessario fare degli ulteriori
ragionamenti. In una situazione di public speaking di tipo business (ma anche
pitch universitari) l'obiettivo è lasciare un impressione positiva agli
interlocutori, in modo tale da agevolare o/e la comunicazione. Partendo
dal presupposto che, come già accennato precedentemente, l'abito fa il
monaco\footnote{Ovvero il contenitore influenza molto il contenuto.}, il
vestiario conta e influenza come una persona appare. La valutazione di oggetti,
idee, persone viene fatta molto velocemente dall'essere umano, ed è una sua
proprietà particolare. Ciò fa si che il 70-80\% dell'impressione che
un ascoltatore si fa di una presentazione avviene nei primi 30 secondi circa.
Ovviamente ciò è un tempo troppo esiguo per valutarne i contenuti. Tale
valutazione viene fatta sulla base di:
\begin{itemize}
\item aspetto del presentatore;
\item contesto;
\item slides o contenuti;
\item comunicazione non verbale.
\end{itemize}

L'estetica di un individuo, quindi, è abbastanza importante. Bisogna sfruttare
questo meccanismo a proprio favore, dato che questa meccanica si applica a tutti
gli umani. Per fare ciò è necessario prima di tutto valutare sé stessi:
presentare in un abito in cui non si è a proprio agio, anche se vestiti in modo
appropriato per l'occasione, può avere un effetto negativo. Ciò non significa
essere pigri e non provare ad abituarsi a portare anche abiti eleganti.

In generale, per farsi ricordare e dare una buona impressione è buona norma
vestirsi un po' meglio di quanto l'occasione
richieda\footnote{\textit{overdressing}}: in un ambiente informale un uomo in
camicia può risaltare, un uomo in frac è fuori luogo. Vestirsi in modo troppo
elegante per il contesto in cui si deve presentare fa perdere di credibilità.

Gli eventi che richiedono un particolare tipo di vestito, per un uomo, sono:
\begin{itemize}
\item \textbf{White tie}: frac/estrema eleganza;
\item \textbf{Black tie}: smoking o completo scuro (soprattutto se la sera,
 dopo le 17);
\item \textbf{Business casual/easy sabaudo}\todo{non so come si scriva}:
 giacca (a volte facoltativa) e jeans oppure spezzato.
\end{itemize}

Anche in un colloquio di lavoro non esiste un vestito ``jolly'' ma è opportuno
vestirsi adeguatamente al ruolo che si vuole andare a ricoprire (eventualmente
overdressed). Quando si lavora invece è meglio vestirsi per il ruolo che si
vuole ricoprire invece che al ruolo ricoperto. È sempre bene
ricordarsi che è bene controllare l'abito che si veste, non il contrario.
