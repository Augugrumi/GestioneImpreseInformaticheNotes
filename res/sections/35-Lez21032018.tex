\subsection{Motivi per fondare una startup}
Ci sono due motivi principali per fondare una startup:
\begin{itemize}
 \item successo
 \item guadagno
\end{itemize}

Gli startupper che mirano alla fama cercano di rimanerne il più tempo possibile
``in sella'' alla startup, rimandando l'exit più tardi possibile (in quanto la
fama calerebbe poi). Questo significa essere più selettivi negli investitori
che vengono scelti e il guadagno è molto minore.

All'inizio di un nuovo progetto è necessario decidere quale strategia adottare:
se massimizzare il risultato del progetto e l'aspetto economico o se
massimizzare la fama personale. In base a ciò vanno scelti determinati
investitori, che dovrebbero condividere la stessa visione, altrimenti si
potrebbero verificare situazioni di conflitto.

\subsection{Due tipi di investitori}

Esistono due tipi di investitori:
\begin{itemize}
 \item finanziario
 \item industriale
\end{itemize}
Nel primo caso, investitori come Business Angels o Venture Capital sono coloro
che mirano alla pura speculazione finanziaria. Gli investitori industriali
invece sono quegli investitori (principalmente aziende) che, selezionando un
numero ristretto di start-up, investe in maniera più selettiva dove trova un
progetto che può essere ritenuto interessante nella loro area di competenza.
Le differenze cruciali tra i due sono:
\begin{itemize}
 \item L'investitore industriale può fornire supporto tecnologico
(partnership). Lo svantaggio è che in questa maniera l'investitore industriale
potrebbe pretendere il controllo della \textit{governance} oppure essere
ingerente in merito
 \item L'investitore finanziario guarda agli aspetti finanziari della cosa,
ovvero solo al ritorno monetario. Viceversa, l'investitore industriale è più
tollerante nelle fluttuazioni del successo che può avere l'operazione, e non è
interessato solamente ai risultati finanziari, ma anche a quello
tecnico/industriale.
 \item L'obiettivo finale dei due investitori è molto diverso. Il finanziario
mira solamente all'exit finale. L'investitore industriale potrebbe non voler
l'exit ma addirittura mira al consolidamento interno del gruppo, ovvero
a ottenere la maggioranza delle quote, facendola entrare nel gruppo aziendale.
\end{itemize}

Per un founder con l'investitore industriale si è già trovato l'exit finale.
Questa strada è ottima come scelta, ma preclude il fatto di avere investori
finanziari. Si ha quindi di fatto, una scelta esclusiva.
È possibile che ci sia anche una soluzione mista, ma questa può portare a
problemi: infatti se gli investori sono misti quelli finanziari si potrebbero
ritrovare con dei capitali bloccati che non saranno in grado di recuperare con
un exit. In questo caso, si rischia che l'investitore finanziario trovi accordi
con quello industriale per farlo arrivare fino al 51\% delle quote. In questo
caso, sarà il fondatore stesso ad avere il suo capitale bloccato. La soluzione
quindi è quella di trovare un accordo di vendita con l'investitore industriale
dove si finisce per vedere la maggior parte delle quote (mantentndo solo, per
esempio, il 5\% totale). In questo caso sarà possibile effettuare l'exit con
maggior successo e capitalizzare.

Come si può valutare la propria startup? Di base non è semplice. Le vendite di
quota delle startup dovrebbero essere ampie all'inizio, per poi diminuire man
mano\footnote{Un andamento contrario può essere visto come un fatto negativo da
parte del mercato.}.

\textbf{Golden paracadute}: quanto il proprietario di un'azienda cambia, è a
totale discrezione dei nuovi ``padroni'' decidere se mantenere il corrente
amministratore delegato o cambiarlo con uno nuovo. Per evitare questo è
importante verificare le clausole, ovvero settare un \textit{paracadute d'oro},
in cui si settano delle condizione di assunzione per tot. anni oppure si
stabilisce una liquidazione cospicua al termine dell'incarico.

\chapter{Brand}

Come nella moda, anche nel mondo della tecnologia esistono i \textbf{trend}
(mode). Esse oscillano e si muovono nel tempo.

Quali sono le difficoltà lato startupper e lato sviluppatori? Sono capire le
necessità del periodo per ottenere dei ritorni. Ci sono diverse modalità per
capire quando investire.

La \textit{Gartner}, azienda multinazionale per l'analisi del mercato e
consulenza aziendale, presenta un approccio orientato alle tecnologie. Tra gli
strumenti che mette a dispozione è presente l'\textbf{hype cycle}.
Si è notato che tutte le tecnologie tendono a percorrere un andamento come
descritto in figura \todo{aggiungere figura slide 3.6 e aggiungerla agli
appunti!}.
Si può vedere sulle $y$ si ha la visibilità che la tecnologia ha, sulle $x$ la
sua maturità. Il grafico è diviso in un percorso in cui tutte le tecnolgie
partono con visibilità 0 per poi arrivare al picco dell'infatuazione in cui
sembra che quella tecnolgia sia necessaria quasi sempre e non se ne possa fare
a meno, per poi arrivare alla delusione delle aspettative stesse. A questo
punto la tecnologia sparisce dal mercato e dai media. A questo punto la
tecnologia cresce di nuovo per raggiungere un certo livello e stablizzarsi.

Alla nascita di una tecnologia la capacità di adozione è molto minore rispetto
alla visibilità che ha.

\paragraph*{Cicli di vita dei prodotti} I prodotti hanno dei specifici cicli di
vita, il cui picco di crescita varia in base alla caratteristica del prodotto
stesso: auto per esempio hanno un ciclo di vita del prodotto più lungo rispetto
a un prodotto tecnologico (ad esempio un'applicazione). La vita di un prodotto
può essere allungata in diverse maniere: una di queste può essere l'utilizzo di
aggiornamenti (nuove versioni, aggiunte di funzionalità).
Un errore comune è associare la visibilità di un prodotto con il suo ciclo di
vita.
\todo{Aggiungere grafico per slide 3.11}

\subparagraph*{Gartner priority matrix} Identifica le varie tecnologie e ci
associa il livello di impatto che la tecnologia può avere sulle persone.

\section{Significato di un brand}

Con Brand si fa riferimento sicuramente al logo aziendale, alla sua immagine
(cui fa parte ad esempio il colore), l'insieme di valori, tutti gli aspetti
connessi a quello che per noi rappresenta il marchio che si sta considerando.

\paragraph*{Terminologia} Un brand presenta i seguenti punti caratteristici:
\begin{itemize}
 \item Brand image: tutto ciò che è connesso al brand tramite l'immagine
 \item Brand identity: ogni brand volontariamente (ovvero cerca) o
spontaneamente (gli vieen associata) ha una determinata identità. Spostare la
brand identity è difficilissimo, richiede un sacco di risorse e investimenti.
Per questo la reputazione è sempre da tenere sott'occhio, tutt'oggi soprattutto
perché con la diffusione di internet e dei social network è facile diffonderne
un'opinione negativa.
 \item Brand reputation: è la reputazione che un brand ha agli occhi degli
utenti. Se la reputazione tempo fare era di tipo \textit{broadcast} ovvero
andava da un punto a tutti, oggi è ``distribuita'': viene discussa e parlata
tra la gente
 \item Brand awareness: quanto è conosciuto un brand. È la consapevolezza che
ciascuno di noi ha di un brand. È uno degli asset più importanti che possono
avere le aziende storiche e ben strutturate. Le aziende serie lavorano che le
\textit{personae}, ovvero con quelle persone di riferimento per il prodotto di
vendita, cercandone determinati valori e caratteristiche. Più si identificano
le caratteristiche più facile è per determinate persone spostare quelle
caratteristiche
 \item Brand value
\end{itemize}
