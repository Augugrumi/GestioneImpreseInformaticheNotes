\paragraph*{Personalizzazione} I clienti possono essere disposti a pagare per
avere un prodotto esattamente ritagliato sulle loro esigenze. Negli anni
recenti si parla spesso di personalizzazione di massa (\textit{mass
customization}) e di co-creazione. Per esempio, la sartoria fa le camicie su
misura, e il loro costo è superiore\footnote{In quanto, rispetto a un'economia
di scala dove una maggiore produzione abbassa i costi, la produzione invece è
singola e ``fatta a mano''.}, come anche il tempo stesso di produzione della
camicia (che è lunga in quanto dev'essere fatta). Qui esistono due tipologie di
clienti: c'è chi vuole la gratificazione immediata dopo l'acquisto e chi invece
preferisce spendere del tempo nella personalizzazione e nell'attesa ma ricevere
un prodotto unico.
Un altro tipico esempio lo si ha nelle auto.

\paragraph*{Supporto} Il valore può venire creato semplicemente aiutando il
cliente ad ottenere un determinato risultato, facendo si che un certo lavoro
venga portato a termine. Alcuni esempi sono le attività di consulenze o aziende
open-source che offrono servizi di consulenza in cambio (ad esempio Red Hat).

\paragraph*{Design} Un prodotto può venire scelto perché si differenzia per un
design superiore. Questa è una delle value proposition centrali della moda
(insieme al brand).

\paragraph*{Brand/Status} A volte il puro utilizzo e l'ostentazione di un
determinato prodotto può creare \textbf{valore} per un cliente. È importante
notare che non solo i prodotti luxury vengono scelti in base al brand, ma anche
per prodotti di massa si può fare leva sul brand per la value proposition.

\paragraph*{Prezzo} Offrire il medesimo servizio ad un costo più basso rispetto
alla concorrenza è un modo usuale di creare valore, come ad esempio Idea,
Ryanair, ecc\dots Lavorare sul prezzo ha una leva molto forte, ma andare a
lavorare a ribasso sul prezzo del prodotto ha un impatto molto forte su tutto
il resto del business model. Quindi è molto complicato offrire lo stesso
livello qualitativo del prodotto rispetto alla concorrenza. È quindi necessario
capire dov'è possibile abbassare la qualità del prodotto su caratteristiche che
non sono rilevate importanti o sensibili per il cliente e puntare invece sulle
altre.

\paragraph*{Riduzione dei costi} Aiutare i clienti a ridurre i loro costi
operativi.\\[0.3cm]

\noindent In passato i clienti cercavano il miglior rapporto qualità/prezzo. In
anni più recenti si è visto un fenomeno molto forte che ho portato
all'inversione di questa curva \todo{Aggiungere immagine slide 5.19} in quanto
c'è stato uno spostamento verso gli estremi. Da un lato si ha una gamma di
prodotti luxury, mentre dall'altro si presentano, per esempio, il proliferare
di compagnie low cost, o di discount.

Questo a causa di un forte cambiamento della realtà di consumo dei singoli.
Riflettendoci attentamente, ciascuno di noi oggi è molto più selettivo
(rispetto al passato) in quanto ognuno di noi ha le idee più chiare su dove uno
vuole spendere i suoi soldi o meno. È quindi chiaro dove in alcune categorie si
fa uso di prodotto low-cost in alcuni settori, mentre in altri si spende molto
di più per prodotti luxury. Le leve per questo fenomeno sono molte, ma
principalmente tutto ciò è dovuto alla crisi economica, che ha ridotto le
capacità di acquisto medio, e dall'altra la diffusione dell'informazione, che
oggi permette di informarsi molto più facilmente su un prodotto.

Quindi, oggi come oggi, posizionare i prodotti in una fascia media è
pericolosissimo, ed è meglio polarizzare quello che si vende negli estremi di
uno delle due fasce.
