\chapter*{Introduzione sul corso}

Il corso varia in base all'interesse degli studenti stessi. Il corso potrebbe
variare da anno in anno.

\paragraph*{Libri di testo} I libri di testo usati sono:
\begin{itemize}
 \item Startup Marketing: errori da evitare e strategie da seguire (A.
Baldissera, B. Bonaventura)
 \item Business Model Generation: a handbook for visionaries, game changers,
and challengers (A. Osterwalder, Y. Pigneur)
\end{itemize}

\paragraph*{Esame} Due metodologie: da frequentanti e non. I primi: prima della
fine della lezione ci sarà un compitino sugli argomenti trattati a lezione, a
cui segue un progetto da fare in gruppo (ideazione e creazione di una
start-up): alla fine del corso ci sarà una presentazione su di essa. L'orale
finale è opzionale, e lo si può fare per migliorare il voto o per mirare alla
lode. Chi frequenta avrà due punti bonus, consegnando il progetto alla prima
sessione d'esame. I gruppi vanno da quattro a sei persone.

Chi non frequenta avrà un esame con orale finale, obbligatorio.

L'esame scritto presenta un gruppo di domande chiuse (a crocette) seguite da
un'insieme di domande aperte. Non vengono rimossi punti se la risposta data è
sbagliata (nelle domande aperte, non per le crocette).

\paragraph*{Ricevimento} Non fa ricevimento, gli si può scrivere una e-mail o
lo si può contattare su Facebook.

\chapter{Introduzione}

Quando si parla del mondo del lavoro spesso si sente parlare di molti termini,
parole. Cominciamo a capire di che cosa stiamo parlando: che cos'è un
imprenditore secondo voi? Si potrebbe parlare di qualcuno che ha i soldi (anche
se non è così), ma anche di qualcuno che da vita a un'attività, ovvero
trasforma dei capitali (investimento iniziale) in un motore di guadagno
economico.

E riguardo a un manager invece? Un manager è qualcuno che gestisce un'attività
che è già avviata. Attività, società, impresa, azienda sono tutti sinonimi, ma
cosa significa? È un'insieme di persone il cui scopo è offrire un
servizio/prodotto dietro guadagno.

\paragraph*{Società e Aziende} Un libero professionista è ha delle conoscenze e
le sfrutta nel mondo del lavoro dietro compenso. Esse sono tutte entità
giuridiche, riconosciuta dallo stato, come erogatrice di determinati
servizi/beni, che genera dei guadagni/flussi di denaro che lo stato tassa. È per
questo che serve il contenitore giuridico.

Quindi infine abbiamo delle organizzazioni che producono qualcosa, per cui
qualcun altro è disposto a pagare per ottenere quel prodotto o quel servizio.
Questo passaggio di denaro viene identificato dallo stato (tramite ad esempio
la partita IVA) per i quali si viene tassati.

Le società solitamente sono delle organizzazioni più strutturate, che impiegano
più individui per produrre cose tendenzialmente più complesse che il singolo
individuo non può fare (e per questo vengono solitamente tassate di più).

I liberi professionisti possono essere:
\begin{itemize}
 \item a chiamata;
 \item ingegneri, Architetti, Idraulico, etc\dots ovvero tutti coloro che
hanno delle competenze specifiche
\end{itemize}
loro si differiscono dal fatto che la società ha bisogno di un gruppo di
persone per produrre/erogare quel singolo servizio. Si differenza dal
professionista che è lui stesso l'erogatore del servizio.

\paragraph*{Imprenditore} Un'imprenditore è un individuo che ha la
proprietà di un'azienda/società. Nelle aziende unipersonali, c'è un individuo
che ha al 100\% la proprietà dell'azienda, ovvero è un qualcosa che si possiede
perché la si ha pagata. La società è una struttura giuridica separata, che non
sono di chi l'ha fondata, ma ne hanno il controllo indiretto magari. I capitali
fanno parte dell'azienda, non di chi la possiede. Un'imprenditore potrebbe
vendere parte di quella struttura per capitalizzare e riguadagnare dei soldi,
altrimenti sono dell'azienda e non dell'imprenditore stesso. L'imprenditore è
sempre e necessariamente proprietario della società.
Le strutture appena si evolvono non sono più di una singola persona, ma fanno
parte di un gruppo di individui o di altre aziende. A questo punto ogni
singolo/azienda ha una percentuale di questa società. Tutti coloro che hanno
una parte della società vengono definiti come consoci, ovvero si ha una
partecipazione all'interno di quella società, indipendentemente che ci lavori o
meno (esistono infatti imprenditori con figure attive o passive).
È importante ricordare che l'imprenditore ha una responsabilità.

\paragraph*{Manager} A differenza dell'imprenditore che ha la proprietà ma
potrebbe non lavorarci dentro, il Manager è una persona che non ha proprietà
sull'azienda, ma ci lavora dentro e la gestisce.
Ad esempio, la FIAT è degli agnelli, che sono gli imprenditori (e non sono
dipendenti della società), ma la gestisce Marchionne, che è un manager e un
dipendente.

\subparagraph*{Differenze tra Manager e Imprenditore}
\begin{itemize}
 \item l'imprenditore non è un dipendente dell'azienda
 \item l'imprenditore, a parte i soci/imprenditori/azionisti, non deve
rispondere a nessun altro
 \item l'imprenditore non ha un contratto da dipendente
 \item l'imprenditore non deve avere competenze particolari per quanto riguarda
l'azienda (questo è vero solo in caso di aziende grandi)
 \item l'imprenditore vive dei ricavi della società/impresa (ne prende una
percentuale), mentre il manager è un dipendente, ha un contratto di lavoro che
non dipende dai ricavi \\[0.5cm] \todo{il prof ha detto che è un concetto
``estremamente interessante''}
\end{itemize}
Fatturato: soldi che ho generato tramite vendite. Il rimanente, ovvero ciò che
rimane dopo le tasse e eventuali spese è definito come l'utile dell'azienda,
che può essere usato per investire ancora nell'azienda o può essere diviso tra
i soci (oppure può essere fatta una scelta intermedia), e vengono definiti come
dividendi.
